\chapter{Opis aplikacije}

Aplikacija opisana u ovom izvještaju prestavlja aplikaciju koja se može koristiti za upravljanje podacima vezanim za jedan aerodrom. Obzirom da sam još prošle godine uzela ovu temu, te da je prethodna verzija bila mnogo skromnije implementirana, odlučila sam da je nadogradim s potrebnim fukncionalnostima. Aplikacija je zamišljena na sljedeći način: osoba (User) koja upravlja aktivnostima na Aerodromu može imati različite uloge (Role) - najčešće su Administrator ili Operator. Osoba može unositi različite Avione (Airplanes), za koje je potrebno unijeti određene specifikacije (između ostalog i kojoj aviokompaniji pripadaju (Airline), te da jedan avion ne smije imati više od 300 sjedala [pretpostavlja se da aerodrom nije toliko veliki da može primiti avione veličine Boeing 777X]). Zatim, mogu se unositi vrste letova (Flight Types), Letovi (Flights), za koje je potrebno unijeti avion koji vrši taj let, na koji izlaz (Gate) putnik treba doći, koja je vrsta leta, vrijeme zauzimanja i oslobađanja piste. Naravno, potrebno je unijeti i putnika, pri čemu se putnik veže za let, te se generiše QR kod kod registriranja Check In-a. Uz putnika je vezana i prtljaga - koja može biti standardna, ručna, te dodatna. Pod standardnom prtljagom, smatra se prtljaga uključena u cijenu, koja ne ide u kabinu s putnikom. Pod ručnom prtljagom, smatra se prtljaga do 10 kg, te dodatnom cijenom. Pod dodatnom prtljagom, smatra se nestandardna prtljaga koju korisnik može imati uz sebe (recimo veća količina novca, metal i slično). \\\\

User ima mogućnost da izgeneriše različite vrste izvještaja: može dobiti izvještaj o svim putnicima, letovima, te svim osobama koje imaju pristup aplikaciji.\\\\

Također, aplikacija je prilagođena različitim podnebljima - postoji mogućnost promjene na osam svjetskih jezika.